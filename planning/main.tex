\documentclass[fleqn,12pt]{scrartcl}
    \usepackage[utf8]{inputenc}
    \usepackage[T1]{fontenc}
    \usepackage{bookman}
    \usepackage[brazil]{babel}
    \usepackage{xcolor}
    \usepackage{amsmath}
    \usepackage{amsfonts}
    \usepackage[margin=2cm]{geometry}
    \usepackage{txfonts}
    \usepackage{xspace}
    \usepackage{url}
    \usepackage{indentfirst}
    
    
    % Minhas definições
    \newenvironment{ct}{\begin{quotation}\color{red!30!black}\sffamily\small Comentário técnico:}{\end{quotation}} % Comentários técnicos
    \newenvironment{cp}{\begin{quotation}\color{green!30!black}\sffamily\small}{\end{quotation}} % Comentários didático-pedagógicos
    \newcommand\ctc[1]{\textcolor{red!30!black}{{\sffamily#1}}} % Comentário técnico curto
    \newcommand\cpc[1]{\textcolor{green!30!black}{{\sffamily#1}}} % Comentário didático-pedagígico curto
    \newcommand\foreign[1]{\textsl{#1}}
    \newcommand\etc{\foreign{etc}}
    \newcommand\proceed{\textcolor{green!50!black}{$\medbullet$}\xspace}
    \newcommand\eg{\foreign{eg}}
    \newcommand\ie{\foreign{ie}}
    \newcommand\condicional[2]{$\lfloor$%
	\textsf{\textcolor{blue}{{\footnotesize #1}}}
	$\leadsto$ #2%
	$\rfloor$}
    \newcommand\forum[1]{\colorbox{yellow}{Discutir no fórum: #1}} 
    \newcommand\parametro[1]{\ensuremath{\langle\text{#1}\rangle}}
    \newcommand\answerfield{\framebox[3cm]{\phantom{A}}~\proceed}
    \newcommand\shortanswerfield{\framebox[2cm]{\phantom{A}}}

    % Título e autor
    \title{Roteiro de produção da AI-0100}
    \subtitle{taxas de variação média e instantânea}
    \author{Dr. Ivan Ramos Pagnossin}
    \date{\today}

%    \hyphenation{vi-zin-han-ça va-ri-á-ve-is}

\begin{document}

    \maketitle
    
    %--------------------
    \section{Orientações}

    \condicional{AI rodando num LMS}{$\langle\text{Nome do aluno}\rangle$,} nesta atividade interativa nós exploraremos os conceitos de taxa de variação média e instantânea de uma função de uma variável ($f: \mathbb{R} \to \mathbb{R}$), escolhida arbitrariamente pelo \foreign{software}. A atividade consiste numa sequência de explicações, questões e instruções, que devem ser executadas na figura interativa acima.

    Você pode seguir esses passos quantas vezes quiser sem valer nota, bem como recomeçá-los a qualquer momento. Quando você achar que já está pronto para ser avaliado, pressione o botão ``valendo nota''. A partir deste momento cada nova tentativa será avaliada, de zero a cem pontos, e deverá ser executada até o fim (o botão ``recomeçar'' não terá efeito). Sua nota será simplesmente a média de todas as suas tentativas. \proceed

    \paragraph{obs.:} para responder às questões, talvez você precise modificar a visualização da janela acima. Par isso, utilize os botões no canto superior esquerdo dela: o botão mais à esquerda (seta) lhe permite arrastar os pontos $A$ e $B$, enquanto o outro botão lhe permite mover o centro da janela e ampliar/reduzir o gráfico. Caso você não esteja vendo os pontos $A$ e $B$, utilize esses botões agora para ajustar a visualização. \proceed

    \begin{ct}
	quando o usuário acessar a atividade interativa pela primeira vez, uma configuração de exercício deve ser sorteada para aquela tentativa. Esta configuração permanece em uso até que o usuário pressione o botão ``recomeçar''.

	Uma ``configuração de exercício'' é composta por uma função $f$, sua derivada $f'$, um $x_A$ e um $x_B$ inicial. 
    \end{ct}

    %-------------------------------
    \section{Taxa de variação média}

    A \emph{taxa de variação média} de uma função $f$ é uma medida de quão rapidamente $f(x)$ varia quando variamos $x$. Dito de outra, se calcularmos $f$ em $x_A$ e depois em $x_B$ (veja a figura), qual será a variação em $f$ associada a esta variação em $x$? Uma forma bastante comum de expressar isso é simplesmente dividir uma variação pela outra:
    \begin{equation*}
	\text{Taxa de variação média de $f$} = \frac{f(x_B) - f(x_A)}{x_B - x_A} = \frac{\Delta f}{\Delta x},
    \end{equation*}
    onde o $\Delta$ é usado para representar uma variação. Ou seja, $\Delta x$ é uma variação em $x$, enquanto $\Delta f$ é uma variação em $f$. $\Delta x$ é arbitrário (de $x_A$ até $x_B$, no caso), pois $x$ é a variável independente, mas $\Delta f$ depende da variação em $x$. \proceed

    A figura acima apresenta o gráfico de $f$, bem como os pontos $A$ e $B$ mencionados no parágrafo anterior. Calcule a taxa de variação média da função neste caso (isto é, calcule a razão $\Delta f/\Delta x$) e escreva seu resultado abaixo. Pressione ``avançar'' para prosseguir.

    \answerfield

    \begin{ct}
	a avaliação deve ser feita usando as coordenadas de $A$ e $B$, questionando o \foreign{applet} através da interface Javascript que ele expõe. A resposta esperada é $(y_B - y_A)/(x_B - x_A)$ e deve haver uma tolerância de 10\% na avaliação da resposta do aluno.

	Exibir os elementos $C$, $\bar{AC}$, $\bar{BC}$ e $\theta$, no \foreign{applet}.
    \end{ct}

    Observe o triângulo $ABC$ na figura. A relação $\Delta f/\Delta x$ é simplesmente a tangente do ângulo $\theta$, ou ainda o coeficiente angular da reta azul.

    %-------------------------------------
    \section{Taxa de variação instantânea}

    Ainda mais importante é a \emph{taxa de variação instantânea} ou \emph{derivada}, que é simplesmente a taxa de variação média de $f$ quando escolhemos $\Delta x$ muito pequeno. Podemos representar esta ideia usando limites, assim:
    \begin{equation*}
	\text{Taxa de variação instantânea} = \lim_{\Delta x \to 0} \frac{\Delta f}{\Delta x} \doteq \frac{df}{dx}.
    \end{equation*}
    
    Para calcular analiticamente a taxa de variação instantânea você precisa da expressão de $f$, que você ainda não tem. Mas é possível estimá-la a partir da figura acima: arraste o ponto $B$ para \textbf{muito} perto de $A$ (use a ferramenta \foreign{zoom}). Este processo simboliza fazer $\Delta x \to 0$. \proceed

    Você pode estimar visualmente a qualidade da sua estimativa comparando a reta azul com a reta cinza. Esta é a \emph{tangente} a $f$ em $x_A$, ou a \emph{inclinação} de $f$ em $x_A$, ou a \emph{taxa de variação instantânea} de $f$ em $x_A$, ou a \emph{derivada} de $f$ em $x_A$.

    Agora faça novamente o cálculo da taxa de variação para $A$ e $B$ muito próximos (você pode errar a resposta em até 10\%):

    \answerfield

    \begin{ct}
	salvar $x_A$, pois será necessário restaurá-lo em seguida.

	A avaliação deve ser feita usando a função derivada de $f$, parte da configuração de exercício. A resposta esperada é $f'(x_A)$, com tolerância de 10\%.
    \end{ct}

    \condicional{Usuário errou}{A resposta correta é \parametro{resposta}.} Perceba que, há pouco, insistentemente escrevi ``em $x_A$''. Isto é importante por que essa quantidade, a taxa de variação instantânea (ou derivada, ou inclinação...) depende do valor de $x$ que escolhemos para calculá-la. Experimente arrastar $A$ (isto é, variar $x_A$) e veja como a reta tangente muda. \proceed

    \begin{ct}
	Restaurar $x_A$ salvo há pouco.
    \end{ct}

    Agora que você estimou a taxa de variação instantânea, posso dizer que função é esta que você está vendo na figura: $f(x) = \parametro{função aleatória}$. \proceed

    Determine --- analiticamente --- a derivada de $f$ para $x$ qualquer e, em seguida, calcule o valor da derivada em $x = x_A$. Escreva o resultado abaixo. Atenção: a tolerância para erros no valor é agora muito menor ($< 1\%$) que no exercício anterior, onde você estava apenas estimando a derivada. Por isso, calcule a derivada analiticamente, isto é, à mão (\textbf{não} faça como você fez no exercício anterior).

    \answerfield

    %------------------------------
    \section{A reta tangente a $f$}

    Se você compreendeu as ideias anteriores, então você já sabe o que é uma derivada. Mas nós ainda não terminamos. Frequentemente é útil saber determinar a expressão da reta tangente a $f$ em $x_A$ (a expressão da reta cinza). Isto pode ser feito através da expressão geral da reta:
    \begin{equation*}
	y(x) - y(x_A) = m (x - x_A),
    \end{equation*}
    onde $m$ é o coeficiente angular da reta. Como estamos interessados na reta tangente a $f$ em $x_A$, $m$ é simplesmente a derivada de $f$ ali, ou o coeficiente angular de $f$ em $x_A$, ou...

    \paragraph{obs.:} a expressão acima pode ser escrita numa forma equivalente que talvez lhe seja mais familiar: $y - y_0 = m(x - x_0)$. Compare-as.
\proceed

    Preencha abaixo os campos com as informações necessárias para representar a reta cinza:

    \begin{equation*}
        y(x) = \shortanswerfield + \shortanswerfield \left(x - \shortanswerfield\,\right)
    \end{equation*}

    \begin{ct}
	as respostas esperadas são: $y_A$, $f'(x_A)$ e $x_A$, respectivamente. Elas têm pesos iguais e tolerância de 1\%.
    \end{ct}

\end{document}

